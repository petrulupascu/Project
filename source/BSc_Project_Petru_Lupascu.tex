\documentclass[a4paper,11pt,oneside]{article}

% To use this template, you have to have a halfway complete LaTeX
% installation and you have to run pdflatex, followed by bibtex,
% following by one-two more pdflatex runs.
%
% Note thad usimg a spel chequer (e.g. ispell, aspell) is generolz
% a very guud ideo.

\usepackage[a4paper,top=3cm,bottom=3cm,left=3cm,right=3cm]{geometry}
\renewcommand{\familydefault}{\sfdefault}
\usepackage{helvet}
\usepackage{parskip}		%% blank lines between paragraphs, no indent
\usepackage[pdftex]{graphicx}	%% include graphics, preferrably pdf
\usepackage[pdftex]{hyperref}	%% many PDF options can be set here
\usepackage{subfigure}
\usepackage{amsmath}

\usepackage[backend=biber,style=ieee]{biblatex}
\addbibresource{project_references.bib}  
\pdfadjustspacing=1		%% force LaTeX-like character spacing

\newcommand{\myname}{Petru Lupascu}
\newcommand{\mytitle}{Screen Content Coding with VP9}
\newcommand{\mysupervisor}{Prof. Dr.-Ing Werner Henkel}
\newcommand{\myssupervisor}{Steffen Schulze(LMI)}

\hypersetup{
  pdfauthor = {\myname},
  pdftitle = {\mytitle},
  pdfkeywords = {},
  colorlinks = {true},
  linkcolor = {blue}
}

\begin{document}
  \pagenumbering{roman}

  \thispagestyle{empty}

  \begin{flushright}
    %%\includegraphics[scale=0.7]{bsc-logo}
  \end{flushright}
  \vspace{20mm}
  \begin{center}
    \huge
    \textbf{\mytitle}
  \end{center}
  \vspace*{4mm}
  \begin{center}
   \Large by
  \end{center}
  \vspace*{4mm}
  \begin{center}
    \Large
    \textbf{\myname}
  \end{center}
  \vspace*{20mm}
  \begin{center}
    \large
    Bachelor Thesis in Electrical and Computer Engineering
  \end{center}
  \vfill
  \begin{flushright}
    \large
    \begin{tabular}{l}
      \mysupervisor\\
      \myssupervisor\\
      \hline
      Name and title of the supervisors \\
      \\
    \end{tabular}
  \end{flushright}
  \vspace*{8mm}
  \begin{flushleft}
    \large
    Date of Submission: May 28th, 2019 \\
    \rule{\textwidth}{1pt}
  \end{flushleft}
  \begin{center}
    \Large Jacobs University --- Focus Area Mobility
  \end{center}
  
  \iffalse

  \newpage
  \thispagestyle{empty}

  With my signature, I certify that this thesis has been written by me
  using only the indicates resources and materials. Where I have
  presented data and results, the data and results are complete,
  genuine, and have been obtained by me unless otherwise acknowledged;
  where my results derive from computer programs, these computer
  programs have been written by me unless otherwise acknowledged. I
  further confirm that this thesis has not been submitted, either in
  part or as a whole, for any other academic degree at this or another
  institution.

  \vspace{20mm}

  Signature \hfill Place, Date

  \newpage
%  \section*{Abstract}
  \begin{abstract}
 
  
  \end{abstract}

  
  \clearpage
  \fi

  \pagenumbering{arabic}


  \newpage
  \tableofcontents

  \newpage
  \listoffigures

  \newpage

  \section{Introduction}
  %backstory
  \indent An Image is a projection of a 3-D scene, characterized by depth, texture and illumination, onto a 2-D plane characterized by texture and illumination 
  without depth information \cite[p.~5]{richardson2002video}. It may be defined as a 2 dimensional signal $ f(x, y) $, where $x$ and $y$ are spatial 
  coordinates and $f$ is the intensity at that point, when $x$, $y$ and $f$ are finite we call this image a Digital Image \cite[p.~1]{gonzalez2008digital}. 
  Therefore, a Video represents a sequence of images over a period of time and can be defined as $f(x,y,t)$, where $x$, $y$, $f$ are spatial and intensity values 
  and $t$ is the time. For the sake of simplicity we will call the two dimensional point a pixel and its intensity pixel value and each image in a video sequence frame. 
  Furthermore, an image can be characterized in terms of its resolution and colour format, for the video, additionally there is length. The resolution
  commonly describes the amount of pixels present in the image, for example: 740x480. The colour format represents a typical arrangemnt of colours in an image such as Grayscale
  where the pixels value represents the light intensity information, commonly 0 to 255 for an 8 bit image. Another important colopur format is the RGB, where image is divided 
  into thee subplanes, each representing the intensity of the Red, Green and Blue light components, for common use cases 8 bits per for each subspace pixel, 24 in total are sufficient.
  Generally, all the parameters mentioned depend on the particular application. However, in most of cases, the ammount of data required to store or transmit a video or an image
  tend to be very large. A two-hour 720x480x24 bits per frame movie, displayed at 30 frames per second must be accessed at $ 31,104,000 \frac{bytes}{s}$ and would require roughly $224 GB$
  of data. \cite[p.~525-526]{gonzalez2008digital}

  



  \indent From the yearly days of digital television, video compression techniques gained increased attention, mainly due to bandwidth always being 
  an expensive asset. Throughout the years, video coding techniques played a vital role in reducing the size of the video sequences without significant
  alteration of its quality. In paralel with advamcement in computer performance, video coding allowed for services such as video telephone and digital 
  televesion to be more accesible, which in turn increased the demand. As a consequence, the developement of video coding techniques was incentivised. 
  Straight Forward Pulse Code Modulation(PCM) was one of the first antepmts on coding video signals at around 140 Mbits/s. Since then video coding 
  have gone a massive progress, modern codecs being able to code Video Signals as low as 9 Mbits/s for HDV format. A newer generation codec targets 
  to achieve the same performance as the previous generation one at the half bitrate. This is done at the expensive of increasing complexity. Most of coding
  techniques require hardware implementations for optimized performance which makes standartization essential to ensure compatibility with as large amount of devices 
  as possible.\cite{ghanbari2011standard}

    %LMI software codec 
    %need to optimize it for screen sharing
    %show issue with artifacts
  \newpage
  \section{Background and literature review}
  %define a video codec
  
  %list the most popular, aka. history

  %software versus hardware codecs in mass communications

  %vp9

  %how does it work

  %compare vp9 out of the box against H family

  %define screene content
    %why it's special

  % go back to the issue
  %why the issue is important
  %the plan for research
  
  
  


  \newpage
  \section*{Bibliography}
  \printbibliography


  \newpage
  \section*{Acknowledgements}
  
  I would like to express my sincere gratitude to my thesis supervisor Prof. Werner Henkel for all the guidance and support throughout the research process and for providing me with all the necessary facilities and equipment. I am also very thankful to Mr. Uwe Pagel for all his technical support and expertise and for his help with extra required materials. Lastly, I am grateful to my parents, Sokol Shyti and Ermira Shyti, and brother, Arbi Shyti, for all their encouragement along the journey and for proofreading my thesis.  


\end{document}
